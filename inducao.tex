\documentclass[a4paper, 10pt]{article}

\usepackage[utf8]{inputenc}
\usepackage[brazilian]{babel}

% The following packages can be found on http:\\www.ctan.org
\usepackage{graphics} % for pdf, bitmapped graphics files
\usepackage{epsfig}   % for postscript graphics files
\usepackage{mathptmx} % assumes new font selection scheme installed
\usepackage{times}    % assumes new font selection scheme installed
\usepackage{amsmath}  % assumes amsmath package installed
\usepackage{amssymb}  % assumes amsmath package installed
\usepackage{makeidx}  % cria indice remissivo \makeindex no preanbulo, \printindex no final do texto para colocação do indice remissivo e \index{} para marcar a palavra que deve ser colocada no indice.

\title{\LARGE \bf O Princípio da Indução e suas Aplicações }

\author{Felipe Luís Pinheiro\footnote{matricula:18/0052667}}

\date{ 18 de novembro de 2018}

\makeindex

\begin{document}
\maketitle

\begin{abstract}

Neste trabalho mostraremos diversas utilizações do princípio de indução em Ciência da Computação.

Começamos discutindo o principio da indução e suas implicações, posteriormente definimos os problemas a serem discutidos e por fim discutimos os problemas propostos até a sua completa solução.

\end{abstract}

\section{Introdução}

%fazer revisão (curta) sobre indução
Indução \index{Indução} é um princípio importante...


\section{Indução Estrutural}
%fazer revisão (curta) sobre indução estrutural

\section{Correção}
%fazer revisão (curta) sonre correção

\section{Problemas}

Nesta seção mostramos os dois problemas a serem discutidos e também mostramos as suas respectivas soluções.

\subsection{Problema 1}

\textbf{Problema 1}: Prove a equivalência entre os princípios da indução forte (PIF) e da indução matemática (PIM)\cite{Apostila}.


\subsection{Problema 2}

\textbf{Problema 2}: Agora utilizaremos o conhecimento adquirido sobre indução para provar a correção de um algoritmo de ordenação de listas conhecido como ``insertion sort'', ou ordenação por inserção. O pseudocódigo deste algoritmo é dado a seguir:

    \begin{equation*}
        \textrm{insertionSort}(l) =
        \left\{
      \begin{array}{ll}
        l, & \textrm{se } l = [] \\
        \textrm{insert}(h, \textrm{insertSort}(l')), & \textrm{se } l = h :: l'
      \end{array}
    \right.
    \end{equation*}
		
onde 

    \begin{equation*}
        \textrm{insert}(x,l) =
        \left\{
      \begin{array}{ll}
        x::[], & \textrm{se } l = [] \\
        x::l, & \textrm{se } l = h::l' \textrm{ e } x \leq h \\
        , & \textrm{ e } x > h 
      \end{array}
    \right.
    \end{equation*}

Nosso objetivo é provar que o algoritmo acima é correto, mas o que isto significa? Como expressar este fato por meio de um teorema? Quais passos intermediários você julga que serão necessários para provar este teorema? Explique e justifique sua solução, e os passos que utilizou para obtê-la, de forma clara e completa por meio de um relatório detalhado\cite{Apostila}.


\section{Conclusão}

\begin{thebibliography}{9}

\bibitem{Apostila}
Flávio L. C. de Moura (Prof.).
\textit{Indução Matemática}.


\end{thebibliography}

\printindex

\end{document}
